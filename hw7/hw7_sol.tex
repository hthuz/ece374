\documentclass[12pt,a4paper]{article}
\usepackage{fullpage}
\pagestyle{plain}
% choose any of the following packages to support AmsTeX
%\usepackage{amsmath,amssymb,amsfonts,mathrsfs,mathptm,bm,mathtools}
% choose the following package to insert eps figures
% for png, jpg or pdf figures, use pdflatex
%\usepackage{graphicx}
\usepackage{amsmath}

%% Presenet code
\usepackage{listings}
\usepackage{color}
\usepackage{float}
\usepackage{graphicx}

\definecolor{dkgreen}{rgb}{0,0.6,0}
\definecolor{gray}{rgb}{0.5,0.5,0.5}
\definecolor{mauve}{rgb}{0.58,0,0.82}

\lstset{frame=tb,
  language=Python,
  aboveskip=3mm,
  belowskip=3mm,
  showstringspaces=false,
  columns=flexible,
  numbers=none,
  numberstyle=\tiny\color{gray},
  commentstyle=\color{dkgreen},
  stringstyle=\color{mauve},
  breaklines=true,
  breakatwhitespace=true,
  tabsize=3
}



\newcommand{\question}[1]{\bigskip\noindent{\textbf{Q{#1} solution}}}

% set HW number
\newcommand{\HWnum}{7}
% specify first and last name and the ID number of students in the group
% append asterix to indicate who is making the submission
\newcommand{\StudentA}{Hanggang Zhu, 3200110457}
\newcommand{\StudentB}{Suhao Wang$^\ast$, 3200110777}
\newcommand{\StudentC}{Lumeng Xu 3200110184}

% ===============================================================
\begin{document}

%%% header
{\noindent \rule{\linewidth}{0.2mm}}\\
\noindent{ECE 374, ZJUI, Spring 2023\hfill%
  \textbf{\large H{}W\HWnum\ Solutions} \hfill \today\smallskip}

\noindent{\hfill \StudentA, \StudentB, and \StudentC \hfill}
\\[-0.2cm]{\noindent \rule{\linewidth}{0.2mm}}
%%% end header

% =============

\question{19}

\question{20}
\\We can try to satisfy the number of f (nodes) by starting with the leaves of the tree.For each node:
\\$parent\_node$ is used to record its parent node, 
\\$children\_edge$ is used to record the edge below the node, 
\\$parent\_edge$ is used to record the edge above it, 
\\$need\_num$ is used to record the number of edges it requires. It will change dynamically.\\


\noindent
For each leaf node, when the number of edges it requires is greater than 0:
\\1. If there is a parent edge, select it and update the arrays.
\\2. If you don't have a parent edge, pick a child edge. Update corresponding arrays.
\\Repeat the preceding steps to update the array of leaf nodes.

\begin{lstlisting}
def minisize(original_Tree, f):
	X = [ ]
	Tree = original_Tree
	leaf_node_list = [ ]
	new_leaf_list = [ ]
	for i in Tree.nodes():
		  	need_num[i] = f(i)
			children_node[i] = # list of all the children nodes of i
			children_edge[i] = # list of all the children edges of i
			parent_node[i]   = # the parent node of i

			if (i is leaf_node):
					leaf_node_list.append(i)

	while (Tree.nodes() is non-empty):
			leaf_node_list = new_leaf_list
			new_leaf_list = [ ]
			for i in leaf_node_list:  
					while need_num[i]>0:
							if (parent_edge[i] is not none):
									dedge = parent_edge[i]
									parent_edge[i] = none
									need_num[parent_node[i]]-=1
									Remove dedge from children_edge[parent[i]] 
							else:
									dedge = # any edge in children_edge[i]
							    	Remove dedge from children_edge[i]
							X.append(dedge)
							need_num[i] -= 1
					Remove i from Tree
					if (parent[i] is not in new_leaf_list):
						new_leaf_list.append(parent[i])
			
	lengthX = len(X)
	return X, lengthX

# notes:
# Tree.nodes() will return the list of all the nodes
# We maintain some arrays to save time

\end{lstlisting}
There are n vertices. Set the arrays needs $n$ time, while loop totally run n time. The size of arrays are in O(1) since degrees are at most 3. Running time is T(n) = O(n). Space used is O(n).

\question{21.A}

\question{21.B}

\question{21.C}

\end{document}