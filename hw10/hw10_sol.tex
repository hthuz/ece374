\documentclass[12pt,a4paper]{article}
\usepackage{fullpage}
\pagestyle{plain}
% choose any of the following packages to support AmsTeX
%\usepackage{amsmath,amssymb,amsfonts,mathrsfs,mathptm,bm,mathtools}
% choose the following package to insert eps figures
% for png, jpg or pdf figures, use pdflatex
\usepackage{amsmath}
\usepackage{graphicx}
\graphicspath{{./img}}
\usepackage{listings}
\usepackage{algorithm}
\usepackage[noend]{algpseudocode}
\algnewcommand{\beginComment}{\textbf{/*}}
\algnewcommand{\endComment}{\textbf{*/}}
\renewcommand{\Comment}[1]{\beginComment~#1~\endComment}


\usepackage{color}
\definecolor{dkgreen}{rgb}{0,0.6,0}
\definecolor{gray}{rgb}{0.5,0.5,0.5}
\definecolor{mauve}{rgb}{0.58,0,0.82}

\lstset{frame=tb,
  language=Python,
  aboveskip=3mm,
  belowskip=3mm,
  showstringspaces=false,
  columns=flexible,
  numbers=none,
  numberstyle=\tiny\color{gray},
  commentstyle=\color{dkgreen},
  stringstyle=\color{mauve},
  breaklines=true,
  breakatwhitespace=true,
  tabsize=3
}

\newcommand{\question}[1]{\bigskip\noindent{\textbf{Q{#1} solution}}}
% set HW number
\newcommand{\HWnum}{10}
% specify first and last name and the ID number of students in the group
% append asterix to indicate who is making the submission
\newcommand{\StudentA}{Hanggang Zhu$^\ast$, 3200110457}
\newcommand{\StudentB}{Suhao Wang, 3200110777}
\newcommand{\StudentC}{Lumeng Xu 3200110184}

% ===============================================================
\begin{document}

%%% header
{\noindent \rule{\linewidth}{0.2mm}}\\
\noindent{ECE 374, ZJUI, Spring 2023\hfill%
	\textbf{\large H{}W\HWnum\ Solutions} \hfill \today\smallskip}

\noindent{\hfill \StudentA, \StudentB, and \StudentC \hfill}
\\[-0.2cm]{\noindent \rule{\linewidth}{0.2mm}}
%%% end header

\question{28.A}
\begin{lstlisting}
use Radix_sort to distinguish the edges
\end{lstlisting}
\question{28.B}

\question{28.C}


\question{29.A}

Since $H - e$ and $G - e$ have the same minimum spanning tree, vertices in MST survivor must be the same as vertices in $G$. Consider these two situations.  

Suppose the edge $e$ we delete from $G$ is an unsafe edge(i.e. it's not in MST of $G$ and it won't affect MST of $G-e$), then the minimum spanning forest of $G-e$ is still a tree. Then in order to have $G-e$ and $H-e$ have the same MST, $H$ must at least contain all safe edges of $G$.

Suppose the edge $e$ we delete from $G$ is a safe edge. If deleting $e$ causes one vertex in $G$ to be disconneted, then the case is the same as above but there will be a minimum spanning forest with multiple trees and $H$ only needs to at least contain all saft edges of G. If deleting $e$ doesn't affect the connected components of $G$, then the MST of $G$ will be different. In order to let $H - e$ have the same minimum spanning forest as $G - e$. $H$ must contain the another connected with the second minimum cost edge that shares the same cut as $e$. These edges can be obtained as MST deleting original MST edges from $G$.

Based on the arguemtn above, the algorithm is as follows.

\begin{algorithm}
  \begin{algorithmic}
    \Function{MSTSurvivor}{$G(V,E)$}
    \State $T \gets MST\ obtained\ by\ Boruvka's\ Algorithm\ on\ G$
    \State $G' \gets T - edges\ in\ T$
    \State $T' \gets MST\ obtained\ by\ Boruvka's\ Algorithm\ on\ G'$
    \State $H \gets (V,edges\ in\ T \cup edges\ in\ T')$
    \State \Return $H$
    \EndFunction
  \end{algorithmic}
\end{algorithm}

The algorithm uses Boruvka's Algorithm twice and some other deletions of edges. Overall, the algorithm takes $O(mlogn)$ time.

\question{29.B}

The idea is to combine several nodes with edges connected between themselives (a subgraph) as one node so that we can get a graph with number of edges is at most $3/2$ of number of vertices. For a subgraph $s$ of $G$ with vertices $v_1,v_2..$, change $s$ to one node of $G$. Edges between $v_i,v_2..$ are ignored and edges like ${v_1,u_1}$, where $u_1$ is not in the subgraph, change it into ${s,u_1}$. If the subgraph can be used with the black box, apply it directly. Otherwise keep forming subgraphs in the subgraph until we can get the MST of the subgraph. The MST of $G$ is formed by connecting MST of different subgraphs using the safe edges. 

By connecting MSTs of subgraphs using the safe edges, we are guarantted to get a MST of $G$, which guaranttes the correctness of the algorithm.

A draft of the algorithm is as follows. 


\begin{algorithm}
  \begin{algorithmic}
    \Function{ComputeMST}{$G(V,E)$}
      \If{$|E| \le 1.5|V|$}
        \State Apply black box to get MST $T$
        \State \Return $T$
      \EndIf
      \State $G_1,G_2...G_k \gets subgraphs\ of\ $G$\ with\ disjoint\ vertices$
      \For{each $G_i$}
        \State $T_i \gets ComputeMST(G_i)$
      \EndFor
      \State $G' \gets metagraph\ with\ nodes\ representing\ T_i\ as\ described\ above$
      \State \Return $ComputeMST(G')$
    \EndFunction
  \end{algorithmic}
\end{algorithm}

The algorithm will iterate through subgraphs of original graph and apply black box to these subgraphs. These subgraphs will be connected. Each edge will be iterated so approximately, the algorithm takes $O(m)$ time


\question{29.C}

To find the MST, apply Kruskal's algorithm using Union-Find data structure. Specifically, 

\begin{algorithm}
  \begin{algorithmic}
    \Function{ComputeMST}{$G(V,E)$} 
    \State Sort edges in $E$ based on cost 
    \State $T \gets empty$
    \State each vertex is a set by itself
    \While{$E$ is not empty}
      \State $e = (u,v) \gets edge\ with\ minimum\ cost$
      \If{$u$ and $v$ are in different sets evaluated using union-find}
        \State $T \gets T \cup e$
        \State Merge sets containing $u$ and $v$ using union-find
      \EndIf
    \EndWhile
    \State \Return T
    \EndFunction
  \end{algorithmic}
\end{algorithm}

Sorting will takes time $O(mlog(m))$ and forming the MST using union find structure will take time $(O(nlog(m)))$. So the running time of the algorithm is $O((m+n)logm)$

\question{30}\\
\textbf{Illustrate:}\\
\textbf{For the meta graph, we preferentially add the sink SCC to the possible pefect set, and delete the edge connected to the SCC to generate the new sink SCCs.}\\
If we add a non-sink SCC to a set, it cannot form a closed set by itself, because some nodes are reachable in other SCCs. If we add a non-sink SCC, we have to add all its downstream SCCs at the same time, which may cause the number of nodes to exceed n and cause us to miss the pefect set (which may actually exist). From a greedy algorithm perspective, we tend to start adding from sink's SCC, always ensuring that it is a closed set.\\
\textbf{correctness:}\\
\textbf{1. For any node A, assuming twin(A) = a, then indegree(A) = outdegree(a), outdegree(A) = indegree(a).}\\
\textbf{2. If the number of removed nodes is less than or equal to n, the removed nodes cannot be twins.}\\
Suppose a group of SCCs have uppercase names and twins SCCs have lowercase names. Suppose "F" is sink SCC, its upstream is ABCD and other SCCs, then f is source SCC, and its downstream is abcd and other SCCs. After we add f, we have to add abcd to add f, the twins(F). Now the amount taken away must be greater than n.\\
\textbf{3. When the number of nodes taken is equal to n, it is closed due to our method of taking away. There are no twins due to note 2. Thus. it is the perfect set.}\\
\textbf{4. When the number of nodes taken is greater than n, it must contain twins according to the pigeonhole principle, so it is not a perfect set.}\\
\begin{lstlisting}
do DFS(G_reverse) and output vertices in decreasing post order.
Mark all nodes as unvisited
Define "result" as the possible perfect set
for each u in the computed order do
	if u is not visited then
		DFS(u)
		Let S_u be the nodes reached by u
		Add S_u to result 

		if (len(result)>n): No perfect set
		else if (len(result)==n): result is the perfect set
		else: continue

		Output S_u as a strong connected component
		Remove S_u as well as the adj edges from G
\end{lstlisting}
runing time is O(m+n), "for each u in the computed order do" and "DFS(u)" could be combined to reduce the running time to O(n+m) rather than O(n(n+m)).\\
the space used is O(n).\\


\end{document}
