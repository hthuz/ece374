\documentclass[12pt,a4paper]{article}
\usepackage{fullpage}
\pagestyle{plain}
% choose any of the following packages to support AmsTeX
%\usepackage{amsmath,amssymb,amsfonts,mathrsfs,mathptm,bm,mathtools}
% choose the following package to insert eps figures
% for png, jpg or pdf figures, use pdflatex
%\usepackage{graphicx}
\usepackage{amsmath}

%% Presenet code
\usepackage{listings}
\usepackage{color}
\usepackage{float}
\usepackage{graphicx}

\definecolor{dkgreen}{rgb}{0,0.6,0}
\definecolor{gray}{rgb}{0.5,0.5,0.5}
\definecolor{mauve}{rgb}{0.58,0,0.82}

\lstset{frame=tb,
  language=Python,
  aboveskip=3mm,
  belowskip=3mm,
  showstringspaces=false,
  columns=flexible,
  numbers=none,
  numberstyle=\tiny\color{gray},
  commentstyle=\color{dkgreen},
  stringstyle=\color{mauve},
  breaklines=true,
  breakatwhitespace=true,
  tabsize=3
}



\newcommand{\question}[1]{\bigskip\noindent{\textbf{Q{#1} solution}}}

% set HW number
\newcommand{\HWnum}{8}
% specify first and last name and the ID number of students in the group
% append asterix to indicate who is making the submission
\newcommand{\StudentA}{Hanggang Zhu$^\ast$, 3200110457}
\newcommand{\StudentB}{Suhao Wang, 3200110777}
\newcommand{\StudentC}{Lumeng Xu 3200110184}

% ===============================================================
\begin{document}

%%% header
{\noindent \rule{\linewidth}{0.2mm}}\\
\noindent{ECE 374, ZJUI, Spring 2023\hfill%
  \textbf{\large H{}W\HWnum\ Solutions} \hfill \today\smallskip}

\noindent{\hfill \StudentA, \StudentB, and \StudentC \hfill}
\\[-0.2cm]{\noindent \rule{\linewidth}{0.2mm}}
%%% end header


% =============

\question{22.A}

Represent the underground tunnel system as a connected undirected graph $G = (V,E)$, where $V$ is the set of intersections of tunnels and $E$ is the set of tunnels. $n$ is size of $V$ and $m$ is size of $E$. To check if a tunnel $e$ is critical, simply check if $G' = (V,E - {e})$ is connected. Use depth first search to determine if $G'$ is connected takes linear time $O(n + m - 1)$.

\begin{lstlisting}
  def isCritical(G(V,E),e):
    T = Tree/Forest genearted by DFS((V, E - {e}))
    if T is a Forest(has multiple root nodes)
      return true
    return False
\end{lstlisting}


\question{22.B}

If tunnel $\{v_i, v_j\}$ is destroyed and cops can't still find a way going from $v_i$ to $v_j$ or from $v_j$ to $V_i$, this shows that there is only one path from $v_i$ to $v_j$ or from $v_j$ to $v_i$, without forming cycles. On other words, $\{v_i, v_j\}$ is critical if it is not an edge of any cycle. To indentify every critical tunnel, do depth first search to find all vertices that are part of a cycle. Then get critical edge from vertices that are not part of any cycle.

\begin{lstlisting}
def FindCriticalTunnel(G(V,E)):
  T(V,E1) = Tree generated by DFS(G)
  CycleVertices = empty
  CriticalEdge = empty
  for edge({vi, vj}) in E - E1 do
    CycleVertices += vi,vj
  for vetex(vi) in V - CycleVertices:
    CriticalEdge += edges in Adj(vi)
  return CriticalEdge
\end{lstlisting}

DFS takes linear time and finding vertices in cycles as well as find critical edges takes linear time. So the total time complexity is $O(n + m)$.

\question{22.C}

The problem is equivalent to find all critical vertices $v$ such that $G' = (V - \{v\}, E - \mbox{\{edges in Adj(v)\}})$ are disconnected. Suppose vertex $v_i$ is removed and $v_i$ is in one cycle $v_1,v_2...v_i,v_{i+1}...v_1$. Then the rest of vertices are still connected using the edge in reverse order. So critical vertices are those that are in one path without forming a cycle. Use similar approach all critical vertices can be found.

\begin{lstlisting}
def FindCriticalIntersetion(G(V,E)):
  T(V,E1) = Tree generated by DFS(G)
  CycleVertices = empty
  CriticalVertices = empty
  for edge({vi, vj}) in E - E1 do
    CycleVertices += vi,vj
  for vetex(vi) in V - CycleVertices:
    for edge({vi,vj}) in Adj(vi)
    CriticalVertices += vi,vj
  return CriticalVertices
\end{lstlisting}

Note that even though some vertices participate in forming into one cycle, if it also participates in forming a non-cycle path, it is critical as well. Similarily, the time complexity is $O(n + m)$


\question{23.A}

\question{23.B}

\question{23.C}

\question{23.D}

\question{24.A}

This problem can be solved by first finding all the strongly connected components of the graph, then reverse the output meta graph, if there is no edges come out of a strongly connected component, it means that 
the initial vertex v can get propagated to all the rebels.

\begin{lstlisting}
def RebelPropagateAll(G(V,E)):
  list $S_{SCC}$
  do DFS($G^{rev}$) and output vertices in decreasing post order
  Mark all nodes as unvisited
  for each v in the computed order do
      if v is not visited then 
           DFS(v)
           Let $S_v$ be the nodes reached by v
           Add $S_v$ to $S_{SCC}$
           Remove $S_v$ from G
  for each $S_v$ in $S_{SCC}^{rev}$ do
      if Adj{$S_v$} = 0
      Output v
      else
      Output "no solution"
\end{lstlisting}

As a rebel can only send messages to one other rebel, the number of edge is no larger than the number of vertex. As the time complexity of computing reverse and DFS or getting through all 
points is all linear, the time complexity of this algrithom is $O(n)$.

\question{24.B}

The algrithom of this problem is almost the same of the problem above, the only difference is that we need to find the number of the strongly connected components which
have no edges come out of it, and output the number.

\begin{lstlisting}
  def MinRebelPropagateAll(G(V,E)):
  list $S_{SCC}$
  num_of_rebel <- 0
  do DFS($G^{rev}$) and output vertices in decreasing post order
  Mark all nodes as unvisited
  for each v in the computed order do
      if v is not visited then 
           DFS(v)
           Let $S_v$ be the nodes reached by v
           Add $S_v$ to $S_{SCC}$
           Remove $S_v$ from G
  for each $S_v$ in $S_{SCC}^{rev}$ do
      if Adj{$S_v$} = 0
      num_of_rebel += 1
      Output num_of_rebel  
\end{lstlisting}

The time complexity of this problem is the same as above, which is $O(n)$.

\question{24.C}

This problem is basically the same as problem A, the difference is that in this problem, a rebel can send messages to many other rebels.

\begin{lstlisting}
  def RebelPropagateAll(G(V,E)):
  list $S_{SCC}$
  do DFS($G^{rev}$) and output vertices in decreasing post order
  Mark all nodes as unvisited
  for each v in the computed order do
      if v is not visited then 
           DFS(v)
           Let $S_v$ be the nodes reached by v
           Add $S_v$ to $S_{SCC}$
           Remove $S_v$ from G
           for each $S_v$ in $S_{SCC}^{rev}$ do
               if Adj{$S_v$} = 0
               Output v
               else
               Output "no solution"
\end{lstlisting}

Assume the number of edges of the original graph is $m$, the time complexity is $O(m+n)$.

\question{24.D}

For this problem, when we make the meta graph of the original graph to a circle, the number of reassignments can be minimized, which is the number of sinks plus the sum of the corresponding source -1 of the meta graph.

\begin{lstlisting}
  def RebelPropagateAll(G(V,E)):
  list $S_{SCC}$
  num_of_reassignments <- 0
  num_of_sink <- 0
  num_of_cor_source <- 0
  do DFS($G^{rev}$) and output vertices in decreasing post order
  Mark all nodes as unvisited
  for each v in the computed order do
      if v is not visited then 
           DFS(v)
           Let $S_v$ be the nodes reached by v
           Add $S_v$ to $S_{SCC}$
           Remove $S_v$ from G
  for each $S_v$ in $S_{SCC}$ do
      if Adj{$S_v$} = 0
      num_of_sink += 1
      for each $S_v$ in $S_{SCC}^{rev}$ do
               if Adj{$S_v$} = 0
               num_of_cor_source += 1
  num_of_reassignments = num_of_sink + num_of_cor_source - 1
  Output num_of_reassignments        
\end{lstlisting}

The time complexity of this algrithom is $O(n)$.

Proof of correctness of this algrithom:
First, we need to delete one edge of the sink strongly connected component, because a rebel can only send a sigle message to one other rebel, there is at least one edge of each vertex. The number of reassignments of this operation is the number of sinks.
Then, we need to connect the sink to another source to make the whole graph a circle, which need the number of corresponding sources - 1 reassignments. So, the minimal number of contacts of a single rebel is the number of the sink plus the number of the corresponding sources - 1.

\end{document}