\documentclass[12pt,a4paper]{article}
\usepackage{fullpage}
\pagestyle{plain}
% choose any of the following packages to support AmsTeX
%\usepackage{amsmath,amssymb,amsfonts,mathrsfs,mathptm,bm,mathtools}
% choose the following package to insert eps figures
% for png, jpg or pdf figures, use pdflatex
\usepackage{amsmath}
\usepackage{graphicx}
\graphicspath{{./img}}
\usepackage{listings}
\usepackage{algorithm}
\usepackage[noend]{algpseudocode}
\algnewcommand{\beginComment}{\textbf{/*}}
\algnewcommand{\endComment}{\textbf{*/}}
\renewcommand{\Comment}[1]{\beginComment~#1~\endComment}


\newcommand{\question}[1]{\bigskip\noindent{\textbf{Q{#1} solution}}}
% set HW number
\newcommand{\HWnum}{11}
% specify first and last name and the ID number of students in the group
% append asterix to indicate who is making the submission
\newcommand{\StudentA}{Hanggang Zhu$^\ast$, 3200110457}
\newcommand{\StudentB}{Suhao Wang, 3200110777}
\newcommand{\StudentC}{Lumeng Xu 3200110184}

% ===============================================================
\begin{document}

%%% header
{\noindent \rule{\linewidth}{0.2mm}}\\
\noindent{ECE 374, ZJUI, Spring 2023\hfill%
	\textbf{\large H{}W\HWnum\ Solutions} \hfill \today\smallskip}

\noindent{\hfill \StudentA, \StudentB, and \StudentC \hfill}
\\[-0.2cm]{\noindent \rule{\linewidth}{0.2mm}}
%%% end header

\question{31.A}

(1)\textbf{Algorithm}: We can just use the brute force method for solving the socially distanced set problem in exponential time by trying all subsets of vertices.
Given the socially distanced set, we can just compare the size to $k$, and return yes/no accordingly.

(2)\textbf{Certificate}: A "proof" in this case would be a subset of vertices $X$ in $G$ where no two vertices of $X$ are connected by an edge, and has a path of length at most 4.

(3)\textbf{Certificate length}: The proof here is a set of $O(n)$ vertices, and can be encoded as a set of $O(n)$ integers. As such, its length is $O(n)$.

(4)\textbf{Verification algorithm}: The verification algorithm for the given proof, would verify that all the vertices in the set $X$ are indeed in the graph, the set $X$ is a distanced set with no two vertices of $X$ are connected by an edge and the length of a path of it is at most 4. Every two paris of vertices in the subset need to be compared. The proof has length $O(n)$ in this case, and the verification algorithm runs in $O(n^2)$ time, if we assume the graph is given to us using adjacent lists representation.

\question{31.B}

(1)\textbf{Algorithm}: We can just use the brute force method for solving the edge independent problem in exponential time by tring all subsets of independent edges.
Given the subset, we can just compare the number of edges of it to $k$, and return yes/no accordingly.

(2)\textbf{Certificate}: A "proof" in this case would be an edge subset $X$ of size $k$ in $G$ where no pair of edges are adjacent.

(3)\textbf{Certificate length}: The proof here is a subset of $O(n)$ edges, and can be encoded as a set of $O(n)$ integers. As such, its length is $O(n)$.

(4)\textbf{Verification algorithm}: The verification algorithm for the given proof, would verify that all the edges in the subset $X$ are indeed in the graph, the set $X$ has $k$ edges with no pair of edges is adjacent. Every two edges in the subset need to be compared.
The proof has length $O(n)$ in this case, and the verification algorithm runs in $O(n^2)$ time, if we assume the graph is given to us using adjacent lists representation.

\question{31.C}

(1)\textbf{Algorithm}: We can just use the brute force method for solving the sum to target problem in exponential time.
Given the subset of the set of positive integers, we can just compare ${\sum_{x\in X}x}$ - ${\sum_{y\in S-X} y}$ to $t$, and return yes/no accordingly.

(2)\textbf{Certificate}: A "proof" in this case would be a subset $X$ $\in S$ where ${\sum_{x\in X}x}$ - ${\sum_{y\in S-X} y}$ = $t$.

(3)\textbf{Certificate length}: The proof here is a set of $O(n)$ elements in the subset, and can be encoded as a set of $O(n)$ integers. As such, its length is $O(n)$.

(4)\textbf{Verification algorithm}: The verification algorithm for the given proof, would verify that the elements in the subset $X$ satisfy ${\sum_{x\in X}x}$ - ${\sum_{y\in S-X} y}$ = $t$.
The proof has length $O(n)$ in this case, and the verification algorithm runs in $O(n)$ time.

\question{31.D}

(1)\textbf{Algorithm}: We can just use the brute force method for solving the 4DM problem in exponential time by trying out all possible subsets.
Given the subset of the set of quadruples, we can just compare the elements in $S$ with $T$ to see if every element of $X\cup Y\cup Z\cup W$ is covered exactly once and return yes/no accordingly.

(2)\textbf{Certificate}: A "proof" in this case would be a subset $S$ $\in T$ where every element of $X\cup Y\cup Z\cup W$ is covered exactly once by one of the quadruples of $S$.

(3)\textbf{Certificate length}: The proof here is a set of $O(4n)$ = $O(n)$ elements in the subset, and can be encoded as a set of $O(n)$ integers. As such, its length is $O(n)$.

(4)\textbf{Verification algorithm}: The verification algorithm for the given proof, would verify that every element of $X\cup Y\cup Z\cup W$ is covered exactly once by one of the quadruples of $S$.
The proof has length $O(n)$ in this case, and the verification algorithm runs in $O(n^4)$ time, if we assume all the four sets are computed.

\question{31.E}

(1)\textbf{Algorithm}: We can just use the brute force method for solving the set disjoinr cover problem in exponential time.
Try all combination of $k$ subsets $\in F$, we can just determine whether they are pairwise disjoint and compute the union of them to see whether they cover $U$ and return yes/no accordingly.

(2)\textbf{Certificate}: A "proof" in this case would be $k$ sets $S_1, ... S_k$ $\in F$ where they are pairwise disjoint and their union covers $U$.

(3)\textbf{Certificate length}: The proof here is a set of $k$ subsets in $F$ and the union size of these subsets is $n$. The legnth is $O(n)$.

(4)\textbf{Verification algorithm}: The verification algorithm for the given proof, would verify that the $k$ sets $S_1, ... S_k$ $\in F$ are pairwise disjoint and their union covers $U$. It needs to compare each two sets and check if they are disjoint. And it also needs to check the union is exactly $U$. The proof has length $O(n)$ in this case, and the verification algorithm runs in $O(n^k)$ time, if we assume all the $n$ elements are contained in each of the $k$ subsets.

\question{32.A}

As described in lecture slides, a clique in graph $G$ is equivalent to an independent set in complement graph $\overline{G}$. So the problem is equivalent to checking if there's at most $k$ independent sets in $\overline{G}$. This problem is equivalent to checking if $\overline{G}$ is $k$-colorable as set of vertices with the same color form an independent set in G. 

The problem now is to to find a polynominal-time reduction from graph k-coloring to SAT. Specifically, given a graph $\overline{G}$ with $n$ vertices and $m$ edges, create a SAT formula $\phi_{\overline{G}}$ such that $\overline{G}$ is k-colorable iff $\phi_{\overline{G}}$ is satisfiable and $\phi_{\overline{G}}$ can be constructed in polynominal time. For each vertex $v_i$, create $k$ variables $v_i.c_1,v_i.c_2,...v_i.c_k$, where $c_k$ corresponds to colors. construct SAT formula as follows.
\begin{equation*}
	\bigwedge_{i=1}^{n} (\bigwedge_{j=1}^{k}(v_i.c_j) \wedge \bigwedge_{a,b=1}^{k} (\overline{v_i.c_a} \vee \overline{v_i.c_b})) \wedge \bigwedge_{\mbox{each edge} (u,v) \in \overline{G}}( \bigwedge_{j=1}^{k}(\overline{u.c_i} \vee \overline{v.c_i}))
\end{equation*}

\noindent
$\bigwedge_{j=1}^{k}(v_i.c_j)$ makes sure each vetex takes one color from $k$ colors. \\
$\bigwedge_{a,b=1}^{k} (\overline{v_i.c_a} \vee \overline{v_i.c_b})$ makes sure each vetex takes only one color instead of multiple colors. \\
$\bigwedge_{\mbox{each edge} (u,v) \in \overline{G}}( \bigwedge_{j=1}^{k}(\overline{u.c_i} \vee \overline{v.c_i}))$ makes sure vertices with adjacent edges don't have the same color.

The construction can be done in $O((n+m)k)$ time. So to reduce CLIQUE-COVER to SAT, first complement graph, which times $O(n + m)$, then reduce it to SAT using the SAT formula given above.


\question{32.B}

No. First of all, this doesn't prove CLIQUE-COVER is in NP. And reduing from CLIQUE-COVER to SAT doesn't prove that CLIQUE-COVER is NP-Hard. One should show that CLIQUE-COVER has a polynominal time cerfiier algorithm and that an proven NPC problem like SAT can be reduced to CLIQUE-COVER.

\question{33}\\
We can prove this problem is NP-hard by a reduction from the Independent set problem.\\
Assume an instance of independent set is a graph $G=(V,E)$ and an integer k. $|V| = n$, $|E| = m$.\\
Let $V\_extra = \{v^e:e \in E(G)\}$, and $|V\_extra|=|E|$. \\
For each edge $(x,y)$ in $E$,connect one vertex to both $x$ and $y$.\\
Then $E\_extra = \{all the new added edge\}$.\\
Construct $G' = (V',E')$ with $V' = V \cup V\_extra$, $E'= E \cup E\_extra $.\\
Now, proving the following statement domonstrates that the triangle free problem is an NPC problem.\\
\textbf{claim: $G'$ has a triangle-free set $S'$, whose size is $k+|E(G)|$, iff G has an independent set S, whose size is  $k$.}\\
\textbf{Step1: if G has an independent set S, whose size is $k$, then $G'$ has a triangle-free set $S'$, whose size is $k+|E(G)|$.}\\
with $|S| >= k$, select $S' = S \cup V\_extra$\\
Since there are no edges between all the pinions in $V\_extra$, there must be edges between the vertices in S in order to form a triangle. But $S$ is an independent set, and there are no edges between them. So $G'$ has a triangle free set $S'$ of size k+|E(G)|.\\


\noindent
\textbf{Step2: G has an independent set S, whose size is $k$, if $G'$ has a triangle-free set $S'$, whose size is $k+|E|$.}\\
Let S' the set for triangle free problem having no triangle and $S'>=|E|+k$, decompose $S'$ into $S_1 = S' \cap V$, $S_2 = S' \cap \{v^e:e \in E(G)\ $ and $e \notin E(S1)\}$.\\
It is obvious that $S_1 \cap S_2 = \emptyset$ and $S=S_1 \cup S_2$.\\
$|S_2|$ can be expressed as $|E(G)|-|E(S_1)|$, which will maximize the S' and does not add triangles. Because if $S_2$ contains a vertex that connect to the edge of $S_1$, there will be a triangle in S'.\\
Thus, $|S| = |S_1|+|S_2|=|S_1|+|E(G)|-|E(S_1)|>=|E(G)|+k $\\
$|S_1|-|E(S_1)|>=k$\\
Remove the edges of S1, the set is an independent set of size $k$.\\
So finding the size of the largest triangle-free subset of vertices in a given undirected graph is NP-hard.


\end{document}
