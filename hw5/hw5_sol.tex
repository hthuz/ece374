\documentclass[12pt,a4paper]{article}
\usepackage{fullpage}
\pagestyle{plain}
% choose any of the following packages to support AmsTeX
%\usepackage{amsmath,amssymb,amsfonts,mathrsfs,mathptm,bm,mathtools}
% choose the following package to insert eps figures
% for png, jpg or pdf figures, use pdflatex
%\usepackage{graphicx}
\usepackage{amsmath}

%% Presenet code
\usepackage{listings}
\usepackage{color}

\definecolor{dkgreen}{rgb}{0,0.6,0}
\definecolor{gray}{rgb}{0.5,0.5,0.5}
\definecolor{mauve}{rgb}{0.58,0,0.82}

\lstset{frame=tb,
  language=Python,
  aboveskip=3mm,
  belowskip=3mm,
  showstringspaces=false,
  columns=flexible,
  basicstyle={\small\ttfamily},
  numbers=none,
  numberstyle=\tiny\color{gray},
  commentstyle=\color{dkgreen},
  stringstyle=\color{mauve},
  breaklines=true,
  breakatwhitespace=true,
  tabsize=3
}



\newcommand{\question}[1]{\bigskip\noindent{\textbf{Q{#1} solution}}}

% set HW number
\newcommand{\HWnum}{5}
% specify first and last name and the ID number of students in the group
% append asterix to indicate who is making the submission
\newcommand{\StudentA}{Hanggang Zhu$^\ast$, 3200110457}
\newcommand{\StudentB}{Suhao Wang, 3200110777}
\newcommand{\StudentC}{Lumeng Xu 3200110184}

% ===============================================================
\begin{document}

%%% header
{\noindent \rule{\linewidth}{0.2mm}}\\
\noindent{ECE 374, ZJUI, Spring 2023\hfill%
  \textbf{\large H{}W\HWnum\ Solutions} \hfill \today\smallskip}

\noindent{\hfill \StudentA, \StudentB, and \StudentC \hfill}
\\[-0.2cm]{\noindent \rule{\linewidth}{0.2mm}}
%%% end header

% =============
\question{13.A}

In the naive Hanoi, disk n need to be moved from src to dest, with n - 1 disks in the tmp peg. But now it is forbidden to directly move disks move src to dest,the disk should be moved to tmp first and then moved to dest. So we need to move disks $n - 1$ to dest first (with intermediate begin on tmp peg), then move disk $n$ to tmp, move disks $n - 1$ from dest back to src(with intermediate being on tmp peg), move disk $n$ to dest and finally move disk $n -1 $ from src to dest again (with intermediate being on tmp peg). The correctness lies in that disk $n$ and disks $n - 1$ will be moved from src to dest, without violating rules.

The pseudo code is as follows:

\begin{lstlisting}
  Hanoi0(n, src, dest, tmp):
    if (n > 0) then
      Hanoi0(n - 1, src, dest, tmp)
      Move disk n from src to tmp
      Hanoi0(n - 1, dest, src, tmp)
      Move disk n from tmp to dest
      Hnaoi0(n - 1, src, dest, tmp)
\end{lstlisting}

The number of moves $T(n)$ can be represented as $T(n) = 3T(n - 1) + 2, n > 1$ with $T(1) = 2$. Solving the recurrence equation, we have $T(n) = 3^n - 1$

% =============

% =============
\question{13.B}
% =============

% =============
\question{13.C}

The largest remaining disk can be removed when nothing is on top of it. This is the same case in naive Hanoi when there's nothing on top largest remaining disk, we can move it to destination. So we can do modification to naive Hanoi with the difference that we remove the largest disk instead of moving it. The removed largest disk won't have any affect as in naive Hanoi the largest disk won't have any affect on rest of disks.

The pseudo code is as follows:
\begin{lstlisting}
  Hanoibyebye(n, src, dest, tmp, max_n):
    if (n > 0) then
      Hanoibyebye(n - 1, src, tmp, dest)
      if (n == max_n) then
        Remove disk n
      else
        Move disk n from src to tmp
      Hanoibyebye(n - 1, tmp, dest, src)
  
\end{lstlisting}

The number of moves $T(n)$ is the same as the case of naive Hanoi, where $T(n) = 2T(n - 1) + 1, n > 1$ with $T(1) = 1$. Solve the recurrence equation and there's $T(n) = 2^n - 1$. Upper bound is $O(2^n)$


% =============
\question{14.A}
% =============

% =============
\question{14.B}
% =============

\question{14.C}


\question{15.A}

\question{15.B}

\end{document}
