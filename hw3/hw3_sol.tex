\documentclass[12pt,a4paper]{article}
\usepackage{fullpage}
\pagestyle{plain}
% choose any of the following packages to support AmsTeX
%\usepackage{amsmath,amssymb,amsfonts,mathrsfs,mathptm,bm,mathtools}
% choose the following package to insert eps figures
% for png, jpg or pdf figures, use pdflatex
%\usepackage{graphicx}

\newcommand{\question}[1]{\bigskip\noindent{\textbf{Q{#1} solution}}}

% set HW number
\newcommand{\HWnum}{3}
% specify first and last name and the ID number of students in the group
% append asterix to indicate who is making the submission
\newcommand{\StudentA}{Hanggang Zhu$^\ast$, 3200110457}
\newcommand{\StudentB}{Suhao Wang, 3200110777}
\newcommand{\StudentC}{Lumeng Xu 3200110184}

% ===============================================================
\begin{document}

%%% header
{\noindent \rule{\linewidth}{0.2mm}}
\noindent{ECE 374, ZJUI, Spring 2023\hfill%
  \textbf{\large H{}W\HWnum\ Solutions} \hfill \today\smallskip}

\noindent{\hfill \StudentA, \StudentB, and \StudentC \hfill}
\\[-0.2cm]{\noindent \rule{\linewidth}{0.2mm}}
%%% end header


\question{7.A}

\question{7.B}

\question{7.C}

\question{7.D}

\question{7.E}



\question{8.A}\\
$\bullet$ \textbf{Intuition:} We can just exchange all the "1" in the L with "$\epsilon$".\\
$\bullet$ \textbf{Answer:}Set NFA $N_1=(\sum_1,Q_1,\delta_1,s_1,A_1)$

	Then NFA accepts DelOnes(L), so we can get:
	
	$\bullet$ $Q_1 = Q$

	$\bullet$ $\delta_1(q,0) = \delta(q,0)$, $\delta_1(q,\epsilon) = \delta(q,1)$, $\delta_1(q,1) = \emptyset $

	$\bullet$ $s_1 = s$

	$\bullet$ $A_1 = A$\\
	

\question{8.B}\\
$\bullet$ \textbf{Intuition:} We can just connect the state machine of "$x$" and the state machine of "$y$". We can add an "$\epsilon$" arrow between the "$x$ state machine's accepting state" and the "$y$ state machine's start state". 
That also means that we can use two DFA M. For one of them, we can swap all the transition function, and swap the start state and the accept state to get an DFA $M_2$. Then, we connect the accept state of $M$ with the new start state of $M_2$ by using $\epsilon$.\\
$\bullet$ \textbf{Answer:}Set $L_2=\{y \ \vline\ y \in L\}$ which is accepted by $M_2$. $M_2=(\sum,Q_2,\delta_2,s_2,A_2)$ Set $L_3=\{x \ \vline\ x \in L\}$ which is accepted by $M_3$. $M_3=(\sum,Q_3,\delta_3,s_3,A_3)$. 

	$\bullet$ $Q_2 = Q_3 = Q$

	$\bullet$  $\delta_2(q,a) = \delta_3(q,a) =\delta(q,a) $

	$\bullet$ $s_2 = s_3 = s$

	$\bullet$ $A_2 = A_3 = A$\\

	Set NFA $N_4=(\sum,Q_4,\delta_4,s_4,A_4)$,then NFA accepts ThereAndBack(L), so we can get:
	
	$\bullet$ $Q_4 = Q \times Q \times \{xs,ys\}$

	$\bullet$  $\delta_4((q,xs),a) = (\delta((q,xs),a),xs)$
 
	$\bullet$	$\delta_4(\delta((q,ys),a),a) = (q,ys)$

	$\bullet$	$\delta_4((a_3,xs)$,$\epsilon) = \{a_2 \ \vline\ a_2 \in A_2\}$, $a_3 \in A_3$

	$\bullet$ $s_4 = s_3$

	$\bullet$ $A_4 =\{s_2\}$\\

\question{8.C}\\
$\bullet$ \textbf{Intuition:} set the state to double state (q,r) so that we can consider x and y.\\
$\bullet$ \textbf{Answer:}Set $L_5=$XOR$(L)$, which is accepted by $M_5$.
Set $M_5=(\sum,Q_5,\delta_5,s_5,A_5)$\\

	$\bullet$ $Q_5 = Q \times Q$

	$\bullet$  $\delta_5((q,r),1) = \{(\delta(q,0),\delta(r,1)),(\delta(q,1),\delta(r,0))\}$, $\delta_5((q,r),0) = \{(\delta(q,1),\delta(r,1)),(\delta(q,0),\delta(r,0))\}$

	$\bullet$ $s_5 = (s, s)$

	$\bullet$ $A_5 = \{(q_1,q_2 \vline\ q_1 \in A$ and $q_2 \in A)\}$\\

\question{8.D}


\question{9.A}

\noindent
Let $F$ be the language $(01)^\ast$. \\
Let $x$ and $y$ be arbitrary strings in $F$. \\
Then $x=(01)^i$ and $y=(01)^j$ for some non-negative integers $i\neq j$. \\ 
Let $z=(10)^i (01)^i$. \\
Then $xz=(01)^i(10)^i(01)^i \in L$. \\ 
And $yz=(01)^j(10)^i(01)^i \notin L$, because $i \neq j$. \\ 
Thus, $F$ is a fooling set for $L$. \\
Because $F$ is infinite, $L$ cannot be regular. \\

\question{9.B}

\noindent
Let $F$ be the language $\{0^{2n}10^n\ \vline\ n \ge 0\}$. \\
Let $x$ and $y$ be arbitrary strings in $F$. \\
Then $x=0^{2i}10^i$ and $y=0^{2j}10^j$ for some non-negative integers $i\neq j$. Assume $j > i$. \\
Let $z=0^i$. \\
Then $xz=0^{2i}10^{2i} \in L$. \\ 
And $yz = 0^{2j}10^{i+j}\notin L$, because $k(i + j) \neq 2j$ for any integar $k$.  \\
Thus, $F$ is a fooling set for $L$. \\
Because $F$ is infinite, $L$ cannot be regular. \\

\question{9.C}

\noindent
Let $F$ be the language $\{a^{2^n}\ \vline\ n \ge 0\}$. \\
Let $x$ and $y$ be arbitrary strings in $F$. \\
As $j=\log_{2}{i}$, $i=2^j$. \\
Then $x=a^{2^i}$ and $y=a^{2^j}$ for some non-negative integers $i\neq j$. \\
Let $z=b^i$. \\
Then $xz=a^{2^{i}}b^i \in L$. \\
And $yz=a^{2^j}b^i\notin L$, because $\log_{2}{2^j} \neq i$. \\
Thus, $F$ is a fooling set for $L$. \\
Because $F$ is infinite, $L$ cannot be regular. \\

\question{9.D}

\noindent
Let $F$ be the language $\{0^{3n^2}\ \vline\ n \ge 0\}$. \\
Let $x$ and $y$ be arbitrary strings in $F$. \\
Then $x=0^{3i^2}$ and $y=0^{3j^2}$ for some non-negative integers $i\neq j$. \\
Let $z=0^{i^2}0^{2i}$. \\ 
Then $xz=0^{4i^2}0^{2i} \in L$. \\
And $yz=0^{3j^2+i^2}0^{2i} \notin L$, because $i \neq j \Rightarrow \sqrt{3j^2 + i^2} \neq 2i$. \\
Thus, $F$ is a fooling set for $L$. \\
Because $F$ is infinite, $L$ cannot be regular. \\

\question{9.E}

\noindent
Let $F$ be the language $a^*c$. \\
Let $x$ and $y$ be arbitrary strings in $F$. \\
Then $x=a^ic$ and $y=a^jc$ for some non-negative integers $i\neq j$. \\
Let $z=d^i$. \\
Then $xz=a^icd^i \in L$. \\
And $yz=a^i cd^j \notin L$, because $i = \#_a(w) \neq j$, where $w$ is $a^i$. \\
Thus, $F$ is a fooling set for $L$. \\
Because $F$ is infinite, $L$ cannot be regular. \\

\end{document}
