\documentclass[12pt,a4paper]{article}
\usepackage{fullpage}
\pagestyle{plain}
% choose any of the following packages to support AmsTeX
%\usepackage{amsmath,amssymb,amsfonts,mathrsfs,mathptm,bm,mathtools}
% choose the following package to insert eps figures
% for png, jpg or pdf figures, use pdflatex
\usepackage{amsmath}
\usepackage{graphicx}
\graphicspath{{./img}}
\usepackage{listings}
\usepackage{algorithm}
\usepackage[noend]{algpseudocode}
\algnewcommand{\beginComment}{\textbf{/*}}
\algnewcommand{\endComment}{\textbf{*/}}
\renewcommand{\Comment}[1]{\beginComment~#1~\endComment}

\usepackage{color}
\definecolor{dkgreen}{rgb}{0,0.6,0}
\definecolor{gray}{rgb}{0.5,0.5,0.5}
\definecolor{mauve}{rgb}{0.58,0,0.82}

\lstset{frame=tb,
  language=Python,
  aboveskip=3mm,
  belowskip=3mm,
  showstringspaces=false,
  columns=flexible,
  numbers=none,
  numberstyle=\tiny\color{gray},
  commentstyle=\color{dkgreen},
  stringstyle=\color{mauve},
  breaklines=true,
  breakatwhitespace=true,
  tabsize=3
}

\newcommand{\question}[1]{\bigskip\noindent{\textbf{Q{#1} solution}}}
% set HW number
\newcommand{\HWnum}{12}
% specify first and last name and the ID number of students in the group
% append asterix to indicate who is making the submission
\newcommand{\StudentA}{Hanggang Zhu$^\ast$, 3200110457}
\newcommand{\StudentB}{Suhao Wang, 3200110777}
\newcommand{\StudentC}{Lumeng Xu 3200110184}

% ===============================================================
\begin{document}

%%% header
{\noindent \rule{\linewidth}{0.2mm}}\\
\noindent{ECE 374, ZJUI, Spring 2023\hfill%
	\textbf{\large H{}W\HWnum\ Solutions} \hfill \today\smallskip}

\noindent{\hfill \StudentA, \StudentB, and \StudentC \hfill}
\\[-0.2cm]{\noindent \rule{\linewidth}{0.2mm}}
%%% end header

\question{34}

For the sake of argument,suppose there is an algorithm DECIDEREVACCEPT that correctly decides the language REVACCEPT. Then we can solve the halting problem as follows
\begin{algorithm}
	\begin{algorithmic}
		\Function{$DECIDEHALT$}{$<M,w>$}
		\State Encode the following Turing machine $M'$:
		\State\quad $M'(x)$:
			\State\quad\quad run $M$ on input $w$
			\State\quad\quad return True
		\If{DECIDEREVACCEPT($<M'>$)}
		\State \Return TRUE
		\Else{}
		\State \Return FALSE 
		\EndIf
		\EndFunction
	\end{algorithmic}
\end{algorithm}

We prove this reduction correct as follows:

$\Longrightarrow$ Suppose $M$ halts on input $w$.
     
	  Then $M'$ accepts every input string $x$.

	  So DECIDEREVACCEPT accepts the encoding $<M'>$.

	  So DECIDEHALT correctly accepts the encoding $<M,w>$.


$\Longleftarrow$ Suppose $M$ does not halt on input $w$.

     Then $M'$ diverges on every input string $x$.

	 So DECIDEREVACCEPT rejects the encoding $<M'>$.

	 So DECIDEHALT correctly rejects the encoding $<M,w>$.

In both cases, DECIDEHALT is correct. But that's impossible, because HALT is undecidable. We conclude that the algorithm DECIDEREVACCEPT does not exist.

\question{35.A}

\question{35.B}

For the sake of argument,suppose there is an algorithm DECIDEACCEPTTURING2 that correctly decides the language ACCEPTTURING2, which is the given language in the question. Then we can solve the halting problem as follows
\begin{algorithm}
	\begin{algorithmic}
		\Function{$DECIDEHALT$}{$<M,w>$}
		\State Encode the following Turing machine $M'$:

		\State\quad $M'(x)$:
		\State\quad\quad run $M$ on input $w$
		\State\quad\quad return True
		
		\If{DECIDEACCEPTTURING2($<M'>$)}
		\State \Return TRUE
		\Else{}
		\State \Return FALSE 
		\EndIf
		\EndFunction
	\end{algorithmic}
\end{algorithm}

We prove this reduction correct as follows:

$\Longrightarrow$ Suppose $M$ halts on input $w$ in space ${|w|}^2$.
     
	  Then $M'$ accepts every input string $x$ in space ${|w|}^2$.

	  So DECIDEACCEPTTURING2 accepts the encoding $<M'>$ in space ${|w|}^2$.

	  So DECIDEHALT correctly accepts the encoding $<M,w>$ in space ${|w|}^2$.


$\Longleftarrow$ Suppose $M$ does not halt on input $w$ in space ${|w|}^2$.

     Then $M'$ diverges on every input string $x$ in space ${|w|}^2$.

	 So DECIDEACCEPTTURING2 rejects the encoding $<M'>$ in space ${|w|}^2$.

	 So DECIDEHALT correctly rejects the encoding $<M,w>$ in space ${|w|}^2$.

In both cases, DECIDEHALT is correct. But that's impossible, because HALT is undecidable. We conclude that the algorithm DECIDEACCEPTTURING2 does not exist.

\question{36.A}

The language SOMETIMESHALT is the same as $A_{Halt}$.

Assume, for the sake of contradiction, that $A_{Halt}$ is decidable. As such, there is a TM, denoted by $TM_{Halt}$, that is a decider for $A_{Halt}$. We can use $TM_{Halt}$ as an implementation of an oracle for $A_{Halt}$, which would imply that one can build a decider for $A_{TM}$. However, $A_{TM}$ is undecidable. A contradiction. It must be that $A_{Halt}$ is undecidable.

So, SOMETIMESHALT is undecidable.
% For the sake of argument,suppose there is an algorithm DECIDESOMETMESHALT that correctly decides the language SOMETMESHALT. Then we can solve the halting problem as follows
% \begin{algorithm}
% 	\begin{algorithmic}
% 		\Function{$DECIDEHALT$}{$<M,w>$}
% 		\State Encode the following Turing machine $M'$:

% 		\State\quad$M'(x)$:
% 		\State\quad\quad run $M$ on input $w$
% 		\State\quad\quad return True
		
% 		\If{DECIDESOMETMESHALT($<M'>$)}
% 		\State \Return TRUE
% 		\Else{}
% 		\State \Return FALSE 
% 		\EndIf
% 		\EndFunction
% 	\end{algorithmic}
% \end{algorithm}

% We prove this reduction correct as follows:

% $\Longrightarrow$ Suppose $M$ halts on input $w$.
     
% 	  Then $M'$ accepts every input string $x$.


% 	  So DECIDEREVACCEPT accepts the encoding $<M'>$.

% 	  So DECIDEHALT correctly accepts the encoding $<M,w>$.


% $\Longleftarrow$ Suppose $M$ does not halt on input $w$.

%      Then $M'$ diverges on every input string $x$.


% 	 So DECIDEREVACCEPT rejects the encoding $<M'>$.

% 	 So DECIDEHALT correctly rejects the encoding $<M,w>$.

% In both cases, DECIDEHALT is correct. But that's impossible, because HALT is undecidable. We conclude that the algorithm DECIDEREVACCEPT does not exist.

\question{36.B}

\question{37.A}

\question{37.B}

\question{37.C}

\question{37.D}

\question{37.E}

\end{document}
