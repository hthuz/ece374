\documentclass[12pt,a4paper]{article}
\usepackage{fullpage}
\usepackage{amsmath}
% \usepackage[
%   backend=biber,
%   style=alphabetic,
%   sorting=ynt
% ]{biblatex}

\pagestyle{plain}
% choose any of the following packages to support AmsTeX
%\usepackage{amsmath,amssymb,amsfonts,mathrsfs,mathptm,bm,mathtools}
% choose the following package to insert eps figures
% for png, jpg or pdf figures, use pdflatex
%\usepackage{graphicx}

\newcommand{\question}[1]{\bigskip\noindent{\textbf{Q{#1} solution}}}

% set HW number
\newcommand{\HWnum}{4}

% specify first and last name and the ID number of students in the group
% append asterix to indicate who is making the submission
\newcommand{\StudentA}{Hanggang Zhu$^\ast$, 3200110457}
\newcommand{\StudentB}{Suhao Wang, 3200110777}
\newcommand{\StudentC}{Lumeng Xu 3200110184}

% ===============================================================
\begin{document}

%%% header
{\noindent \rule{\linewidth}{0.2mm}}
\noindent{ECE 374, ZJUI, Spring 2023\hfill%
  \textbf{\large H{}W\HWnum\ Solutions} \hfill \today\smallskip}

\noindent{\hfill \StudentA, \StudentB, and \StudentC \hfill}
\\[-0.2cm]{\noindent \rule{\linewidth}{0.2mm}}
%%% end header

\question{10.A}

\question{10.B}


\question{10.C}

\question{11.A}
\\First, consider the special case where $i=0$.
\\Then, this language can be written as $L=\{b^j c^k d^l | j,k,l \geq 0$ and $j+l=k\}$.
\\And we get the CFL:
\begin{center}
  $S \rightarrow AB$
  \\$A \rightarrow bAc | \epsilon$
  \\$B \rightarrow cBd | \epsilon$
\end{center}
Then, we consider another special case where $j=i$.
\\Then, we get $k=l$. And this language can be written as $L=\{a^j b^j c^k d^k | j,k \geq 0\}$.
\\And we get the CFL:
\begin{center}
  $S \rightarrow AB $
\\$A \rightarrow aAb | \epsilon$
\\$B \rightarrow cBd | \epsilon$
\end{center}
We found that in both cases, it has the equation $B \rightarrow cBd | \epsilon$, so we guess that in general case, 
we also have this equation.
\\
\\By induction, we get the general case:
\begin{center}
  $S \rightarrow ABC $
\\$A \rightarrow aAb | \epsilon$
\\$B \rightarrow bBc | \epsilon$
\\$C \rightarrow cCd | \epsilon$
\end{center}
To confirm its validity, we found that the number of $a$ plus the number of $c$ is 
equal to the number of $b$ plus the number of $d$, which satisfie the requirement. 
\\So, we conclude this answer. 
\\
\question{11.B}
\\First, we consider the case when only $\omega_i = a$.
\\In this case, the string only has $a$, so it must satisfy the condition 
for $L_B$.
\\Then, we consider the case $\omega_i$ is a balanced string or $\omega_i = a$. 
\\By definition, 
$\omega$ satisfies the condition $\#_a(\omega) \geq \#_b(\omega)$. When $\#_a(\omega)=\#_b(\omega)$,
we can know that $\omega$ can be at least divided to one balanced string $\omega_i$. 
When $\#_a(\omega) > \#_b(\omega)$, we can split the string to at least one balanced string and 
several $a$s. So, it satisfies this condition.
\\
\question{11.C}
\\For the above condition, we conclude that the CFG can be expressed as:
\begin{center}
  $S \rightarrow abS | aSb | Sab | SS | \epsilon $
\end{center}
For $S \rightarrow abS $, $a$ always appears before $b$, so,it satisfies $L_B$.
\\For $S \rightarrow aSb $, it also satisfies that $a$ always appears before $b$.
\\For $S \rightarrow Sab $, $a$ always appears before $b$.
\\For $S \rightarrow SS $, no matter what the value of $S$ is, it always satisfies
$a$ always appears before $b$, which satisfies $L_B$.
\\For $S \rightarrow \epsilon $, the empty string does not change the number of $a$ and
$b$.
\\So, the correctness of the CFG is proved.
\\
\question{12.A}

\question{12.B}

According to the theorem {\em Every context free language is equivalent to a grammer in Chomsky normal form}\cite{model}. Transform context free language $L$ into its Chomsky normal form. In Chomsky normal form each production rule is of the form $A\rightarrow BC \mbox{ or } A\rightarrow a$ besides those containing starting non-terminal $S$. In all cases, each single terminal symbol on the right of production rule represents a character of the langauge. If we add production rule $A\rightarrow \epsilon$ to every production rule of the form $A\rightarrow a$, the new context-free language represents subsequence($L$). So subsequence($L$) is also a context-free language.


\bibliographystyle{plain}
\bibliography{ref}


\end{document}
