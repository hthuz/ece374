\documentclass[12pt,a4paper]{article}
\usepackage{fullpage}
\pagestyle{plain}
% choose any of the following packages to support AmsTeX
%\usepackage{amsmath,amssymb,amsfonts,mathrsfs,mathptm,bm,mathtools}
% choose the following package to insert eps figures
% for png, jpg or pdf figures, use pdflatex
\usepackage{amsmath}
\usepackage{graphicx}
\graphicspath{{./img}}
\usepackage{listings}
\usepackage{algorithm}
\usepackage[noend]{algpseudocode}
\algnewcommand{\beginComment}{\textbf{/*}}
\algnewcommand{\endComment}{\textbf{*/}}
\renewcommand{\Comment}[1]{\beginComment~#1~\endComment}


\newcommand{\question}[1]{\bigskip\noindent{\textbf{Q{#1} solution}}}
% set HW number
\newcommand{\HWnum}{11}
% specify first and last name and the ID number of students in the group
% append asterix to indicate who is making the submission
\newcommand{\StudentA}{Hanggang Zhu$^\ast$, 3200110457}
\newcommand{\StudentB}{Suhao Wang, 3200110777}
\newcommand{\StudentC}{Lumeng Xu 3200110184}

% ===============================================================
\begin{document}

%%% header
{\noindent \rule{\linewidth}{0.2mm}}\\
\noindent{ECE 374, ZJUI, Spring 2023\hfill%
	\textbf{\large H{}W\HWnum\ Solutions} \hfill \today\smallskip}

\noindent{\hfill \StudentA, \StudentB, and \StudentC \hfill}
\\[-0.2cm]{\noindent \rule{\linewidth}{0.2mm}}
%%% end header

\question{31.A}

\question{31.B}

\question{31.C}

\question{31.D}

\question{31.E}

\question{32.A}

As described in lecture slides, a clique in graph $G$ is equivalent to an independent set in complement graph $\overline{G}$. So the problem is equivalent to checking if there's at most $k$ independent sets in $\overline{G}$. This problem is equivalent to checking if $\overline{G}$ is $k$-colorable as set of vertices with the same color form an independent set in G. 

The problem now is to to find a polynominal-time reduction from graph k-coloring to SAT. Specifically, given a graph $\overline{G}$ with $n$ vertices and $m$ edges, create a SAT formula $\phi_{\overline{G}}$ such that $\overline{G}$ is k-colorable iff $\phi_{\overline{G}}$ is satisfiable and $\phi_{\overline{G}}$ can be constructed in polynominal time. For each vertex $v_i$, create $k$ variables $v_i.c_1,v_i.c_2,...v_i.c_k$, where $c_k$ corresponds to colors. construct SAT formula as follows.
\begin{equation*}
	\bigwedge_{i=1}^{n} (\bigwedge_{j=1}^{k}(v_i.c_j) \wedge \bigwedge_{a,b=1}^{k} (\overline{v_i.c_a} \vee \overline{v_i.c_b})) \wedge \bigwedge_{\mbox{each edge} (u,v) \in \overline{G}}( \bigwedge_{j=1}^{k}(\overline{u.c_i} \vee \overline{v.c_i}))
\end{equation*}

\noindent
$\bigwedge_{j=1}^{k}(v_i.c_j)$ makes sure each vetex takes one color from $k$ colors. \\
$\bigwedge_{a,b=1}^{k} (\overline{v_i.c_a} \vee \overline{v_i.c_b})$ makes sure each vetex takes only one color instead of multiple colors. \\
$\bigwedge_{\mbox{each edge} (u,v) \in \overline{G}}( \bigwedge_{j=1}^{k}(\overline{u.c_i} \vee \overline{v.c_i}))$ makes sure vertices with adjacent edges don't have the same color.

The construction can be done in $O((n+m)k)$ time. So to reduce CLIQUE-COVER to SAT, first complement graph, which times $O(n + m)$, then reduce it to SAT using the SAT formula given above.


\question{32.B}

No. First of all, this doesn't prove CLIQUE-COVER is in NP. And reduing from CLIQUE-COVER to SAT doesn't prove that CLIQUE-COVER is NP-Hard. One should show that CLIQUE-COVER has a polynominal time cerfiier algorithm and that an proven NPC problem like SAT can be reduced to CLIQUE-COVER.

\question{33}



\end{document}
